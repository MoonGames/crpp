\part{Commands}
\label{commands}

Tato kapitola popisuje jednotlivé příkazy, které mohou být jako message posílány, jejich strukturu, způsob zpracování a smysl.

Příkazy se dělí na příchozí (posílá je server na klienty), odchozí (klienti je posílají na server) a obousměrné (posílají se na server i na klienta). V části \ref{commands.list_of_commands} je seznam všech příkazů se stručným popisem a v další části je podrobně popsán jednotlivě každý příkaz.

\chapter{List of commands}
\label{commands.list_of_commands}

\section{Outgoing commands}

\begin{itemize}
    % connection
    \item GCIN -- get connection info -- žádost o informaci o připojení (jestli je klient nebo není autentizován)
    \item AUTN -- autenticate -- autentikuje klienta
    % rooms, etc
    \item GRLI -- get rooms list -- žádost o seznam místností na serveru
    \item CROM -- create room -- žádost o vytvoření místnosti
    \item JROM -- join to room outgoing -- žádost o připojení do místnosti
    \item DROM -- disconnect from room outgoing -- žádost o odpojení z místnosti
    % other services
	\item HTIM -- make HTTP image -- žádost o HTTP obrázek
	\item CHAT -- chat message outgoing -- odešle chat zprávu
\end{itemize}

\section{Incomming commands}

\begin{itemize}
    % connection
    \item SINF -- server info -- základní informace o připojení, včetně případného vyžádání autentikace
    \item CINF -- connection info -- informace o stavu připojení (jestli je nebo není uživatel autentizován)
    \item SSUC -- client successfully connected -- informace, že byl klient úspěšně připojen
    \item SERR -- client unsuccessfully connected -- informace, že klient nebyl úspěšně připojen
    % rooms, etc
    \item RLIS -- rooms list -- seznam místností na serveru
    \item JROM -- join to room incomming -- byl jsi připojen do místnosti
    \item DROM -- disconnect from room incomming -- byl jsi odpojen zmístnosti
    \item ULIS -- users list -- seznam uživatelů v místnosti
    % in room painting
	\item LORD -- layers order
	% other services
	\item CHAT -- chat message incomming -- přišla chat zpráva
\end{itemize}

\section{Duplex commands}

\begin{itemize}
    % user
    \item SNIC -- set nick -- nastaví nick uživately

	\item PANT -- paint -- informuje o novém zakreslení/mazání ve vrstvě 
	\item LNAM -- set layer name
\end{itemize}




%\chapter{Command human readable name}

Short command description.

\section{List of parameters}

% here is list of all parameters (names and human readable names)

\begin{itemize}
    % 4 ASCII chars - human readable name
	\item ASC1 -- First parameter
	\item ASC2 -- Second parameter
\end{itemize}

\section{Server action}

This section is mentioned only if its also outgoint command. This text describes what will server do after receved this command.

\section{Client action}

This section is mentioned only if its also incomming command. This text describes what will client do after receved this command.

\section{Detailed description of parameters}

\subsection{First parameter}

Description of first parameter. Data type and connotation.

\subsection{Second parameter}

Bla bla bla!

\chapter{Paint}

Tento příkaz informuje o aktualizaci kreslícího plátna. 

\subsection{List of parameters}

\begin{itemize}
	\item UDTY -- update type
	\item UDID -- update ID
	\item LYID -- layer ID
	\item CNID -- canvas ID
	\item XCOR -- X coordinate
	\item YCOR -- Y coordinate
	\item UIMG -- update image				
\end{itemize}

\subsection{Server action}

Server zakreslí update na příslušné místo a podé rozešle the same update (paint) to all connected users.

\subsection{Client action}

\subsection{Detailed description of parameters}

\subsubsection{Update type}

Update type je typ updatu a obsahuje čtyřbytové neznaménkové celočíselné číslo udávající typ updatu. Může nabívat hodnot $0$ a $1$, kde $0$ reprezentuje přidávácí update a $1$ mazací.

V případě přidávacího updatu se standartním (source over) algoritmem update nakreslí přes obrázek vrstvy. 

V případě odebíracího updatu se jen mění velikost alpha kanálu jednotlivých pixelů updatované vrstvy. Výsledná alpha každého pixelu $D_{a}$ se spočítá jako $D_{a} = S_{a} \cdot P_{a}$ kde $S_{a}$ značí alphu pixelu v updatovacím obrázku a $P_{a}$ značí alphu v v obrázku vrstvy před updatem. Všechny alphy ve výpočtu nabívají hodnot $0$ až $1$, takže je před výpočtem potřeba je převést ze standartní podoby $0$ až $255$ na tyto dělením číslem $255$ a po výpočtu převést zpátky násobením.

\subsubsection{Update ID}

Obsahuje čtyřbytové neznaménkové celé číslo, které identifikuje tento update.

\subsubsection{Layer ID}

Obsahuje čtyřbytové neznaménkové celé číslo, které identifikuje updatovanou vrstvu.

\subsubsection{Canvas ID}

Obsahuje čtyřbytové neznaménkové celé číslo, které identifikuje updatované kreslící plátno (je tak možné kreslit na několik pláten).

\subsubsection{X coordinate}

Obsahuje čtyřbytové neznaménkové celé číslo, které udává X souřadnici levého horního rohu updatu (kam se aplikuje updatovací obrázek).

\subsubsection{Y coordinate}

Obsahuje čtyřbytové neznaménkové celé číslo, které udává Y souřadnici levého horního rohu updatu (kam se aplikuje updatovací obrázek).

\subsubsection{Update image}

Viz část specifikaci \ref{connection.data_types.image}.
\section{Add layer}

Přidá vrstvu do plátna. Obsahuje tyto bloky:

\begin{itemize}
	\item LPOS -- layer position
	\item LNAM -- layer name
	\item CNID -- canvas ID
\end{itemize}

\subsubsection{Server action}

Server vytvoří novou prázdnou vrstvu a zařadí jí do správné pozice. Poté odešle všem připojeným klientům nové layers order a pak name of this layer (set layer name).

\subsubsection{Blocks}

\paragraph{Layer position}

Obsahuje celočíselné čtyřbytové bezznaménkové číslo, které udává kolikátá v pořadí bude vrstva od spoda po přidání. Vrstva na pozici $0$ bude pod všemi ostatními. V případě, že je číslo větší než počet všech vrstev, bude vrstva přidána na vrch.

\paragraph{Layer name}

Obsahuje text s názvem vrstvy.

\paragraph{Canvas ID}

Obsahuje celočíslené bezznaménkové čtyřbytové číslo s ID plátna, do kterého má být vrstva přidána.