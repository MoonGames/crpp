\part{Commands}
\label{commands}

\chapter{Commands}
\label{text.commands}

Tato kapitola popisuje jednotlivé příkazy, které mohou být jako message posílány, jejich strukturu, způsob zpracování a smysl.

Příkazy se dělí na příchozí (posílá je server na klienty), odchozí (klienti je posílají na server) a obousměrné (posílají se na server i na klienta).

Seznam příkazů:

\begin{itemize}
	\item PANT -- paint
	\item LORD -- layers order
	\item ADDL -- add layer
	\item LNAM -- set layer name
\end{itemize}

\section{Příchozí}

\section{Odchozí}

\subsection{Add layer}

Přidá vrstvu do plátna. Obsahuje tyto bloky:

\begin{itemize}
	\item LPOS -- layer position
	\item LNAM -- layer name
	\item CNID -- canvas ID
\end{itemize}

\subsubsection{Server action}

Server vytvoří novou prázdnou vrstvu a zařadí jí do správné pozice. Poté odešle všem připojeným klientům nové layers order a pak name of this layer (set layer name).

\subsubsection{Blocks}

\paragraph{Layer position}

Obsahuje celočíselné čtyřbytové bezznaménkové číslo, které udává kolikátá v pořadí bude vrstva od spoda po přidání. Vrstva na pozici $0$ bude pod všemi ostatními. V případě, že je číslo větší než počet všech vrstev, bude vrstva přidána na vrch.

\paragraph{Layer name}

Obsahuje text s názvem vrstvy.

\paragraph{Canvas ID}

Obsahuje celočíslené bezznaménkové čtyřbytové číslo s ID plátna, do kterého má být vrstva přidána.

\section{Obousměrné}

\subsection{Paint}

Tento příkaz informuje o aktualizaci kreslícího plátna. Obsahuje tyto bloky:

\begin{itemize}
	\item UDTY -- update type
	\item UDID -- update ID
	\item LYID -- layer ID
	\item CNID -- canvas ID
	\item XCOR -- X coordinate
	\item YCOR -- Y coordinate
	\item UIMG -- update image data				
\end{itemize}

\subsubsection{Server action}

Server zakreslí update na příslušné místo a podé rozešle the same update (paint) to all connected users.

\subsubsection{Blocks}

\paragraph{Update type}
Update type je typ updatu a obsahuje čtyřbytové neznaménkové celočíselné číslo udávající typ updatu. Může nabívat hodnot $0$ a $1$, kde $0$ reprezentuje přidávácí update a $1$ mazací.

V případě přidávacího updatu se standartním (source over) algoritmem update nakreslí přes obrázek vrstvy. 

V případě odebíracího updatu se jen mění velikost alpha kanálu jednotlivých pixelů updatované vrstvy. Výsledná alpha každého pixelu $D_{a}$ se spočítá jako $D_{a} = S_{a} \cdot P_{a}$ kde $S_{a}$ značí alphu pixelu v updatovacím obrázku a $P_{a}$ značí alphu v v obrázku vrstvy před updatem. Všechny alphy ve výpočtu nabívají hodnot $0$ až $1$, takže je před výpočtem potřeba je převést ze standartní podoby $0$ až $255$ na tyto dělením číslem $255$ a po výpočtu převést zpátky násobením.

\paragraph{Update ID}
Obsahuje čtyřbytové neznaménkové celé číslo, které identifikuje tento update.

\paragraph{Layer ID}
Obsahuje čtyřbytové neznaménkové celé číslo, které identifikuje updatovanou vrstvu.

\paragraph{Canvas ID}
Obsahuje čtyřbytové neznaménkové celé číslo, které identifikuje updatované kreslící plátno (je tak možné kreslit na několik pláten).

\paragraph{X coordinate}
Obsahuje čtyřbytové neznaménkové celé číslo, které udává X souřadnici levého horního rohu updatu (kam se aplikuje updatovací obrázek).

\paragraph{Y coordinate}
Obsahuje čtyřbytové neznaménkové celé číslo, které udává Y souřadnici levého horního rohu updatu (kam se aplikuje updatovací obrázek).

\paragraph{Update image data}
Obsahuje binární data PNG obrázku typu RGBA (32 bites). Viz http://www.w3.org/Graphics/PNG/. 